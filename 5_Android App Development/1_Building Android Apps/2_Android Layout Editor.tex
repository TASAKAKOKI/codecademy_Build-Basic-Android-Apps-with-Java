⭐❇️➡️✔️➖🔗💎👉👈
Android Layout Editor
        Let's dig into the Layout Editor to understand how we can use it to visualize and modify our application designs

        Without layout files, we have nothing to present to the user. 👉Layout files group View objects together to form cohesive interface elements such as an item in a scrolling list or the collective content visible on-screen👈. But before these layouts reach a user’s fingertips, it’s our job to make sure they fulfill our design vision.

        Thankfully, Android Studio provides a tool that helps us visualize and modify our designs before we deploy them to our application. 👉The Layout Editor interprets the components and their positions, colors, images, fonts, and text strings defined in our layout files and presents them as they would appear on real-world devices.👈

        Let’s explore the Layout Editor and how it helps achieve our design goals.
            🔗https://www.youtube.com/watch?v=x-Tu5cREfZE

        Review:
            👉Android project files are divided into one of three categories: 💎code, 💎resource, and 💎configuration.
            👉We categorize layouts, colors, text strings, images, and beyond as 💎‘resource’ files.
            👉The Layout Editor opens automatically when editing layout files.
            👉The 💎Text tab edits the underlying XML file directly, and the 💎Design tab provides a drag-and-drop interface for achieving the same results.
            👉Attributes define the appearance and behavior of every View and ViewGroup.
            👉In Design mode, the Layout Editor presents every possible attribute that we may supply.
            👉Only two attributes are mandatory: layout_width and layout_height, and the most common values for these attributes are match_parent and wrap_content.
        
        The Layout Editor provides an exact representation of our layout files as they would appear on a real device…
            True
            ✔️False ==> It is only a near approximation

        What purpose do attributes serve in a layout file?
            Leave comments about the View for other developers to read
            Set the order in which View’s appear on screen
            Set the View’s render priority
            ✔️Modify the View’s appearance and characteristics

        Choose the attribute values that will force TextView to grow as wide as its parent but only as tall as its content requires.
            <?xml version="1.0" encoding="utf-8"?>
            <androidx.constraintlayout.widget.ConstraintLayout xmlns:android="http://schemas.android.com/apk/res/android"
                xmlns:app="http://schemas.android.com/apk/res-auto"
                android:layout_width="match_parent"
                android:layout_height="match_parent">
            
                <TextView
                    android:layout_width="👉match_parent👈"
                    android:layout_height="👉wrap_content👈"
                    android:src="@drawable/ic_hello_world"
                    app:layout_constraintBottom_toBottomOf="parent"
                    app:layout_constraintEnd_toEndOf="parent"
                    app:layout_constraintStart_toStartOf="parent"
                    app:layout_constraintTop_toTopOf="parent" />
            
            </androidx.constraintlayout.widget.ConstraintLayout>