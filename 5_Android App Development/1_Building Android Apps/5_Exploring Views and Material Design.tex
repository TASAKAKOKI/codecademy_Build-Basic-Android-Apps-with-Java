⭐❇️➡️✔️➖🔗💎👉👈
Exploring Views and Material Design
        Finding new components, navigating to their documentation, and applying them

        💎LinearLayout, 💎ConstraintLayout, 💎ImageView, and 💎TextView — suffice it to say, that’s not enough. The Android SDK combined with AndroidX offers a wide array of Views and ViewGroups we may incorporate to bring our design vision into reality.

        From buttons to scrolling containers to text input elements, there’s so much more out there. Let’s learn how to find new components, navigate instantly to their documentation, and apply them while following best practices (e.g. Material Design guidelines). Afterward, we will have a better understanding of which components are available to us and how to begin using them in our applications.
            🔗https://www.youtube.com/watch?v=UNvOqlP1VEg

        Takeaways
            👉The component palette helps us find and add a dynamic range of Views, ViewGroups, and more directly to our layouts.
            👉Styles are external resources that define a set of attributes we may apply to any View using the style attribute, e.g. android:style="@style/my_cool_textview".
            👉The CardView ViewGroup and many other components in the palette are inspired by Material Design guidelines.
            👉Google introduced Material Design in 2014 to help standardize the appearance of web and mobile applications across the Google ecosystem.
            👉Material Design continues to evolve and is available for iOS, Android, and several web frameworks—however, applying its principles is entirely optional.
            👉By right-clicking any element within the component palette, we may navigate to its respective documentation or Material Design guidelines.
            👉We may use an include component to embed the contents of an external layout file into our design.

        Which of the following is a false statement regarding Android styles?
            ✔️Styles must feature at least one attribute.
                Android does not require styles to feature any attributes (technically), but that wouldn’t make for a very useful style, now would it?
            Many styles are made available through the Android SDK.
            Some Views inherit a default style from the Android SDK.
            We can create custom styles for our application.
            We can apply them to any View using the style attribute.

        Which of the following is true of CardView?
            CardView is not a ViewGroup.
            CardView has no basis in Material Design.
            ✔️CardView is only accessible through the AndroidX library.
                Right! As the Android ecosystem grew, Google designed new components as optional add-ons rather than bundled with every installation of Android—therefore, CardView is available exclusively through AndroidX.
            CardView is accessible through standard Android APIs.

        When you encounter a new View or ViewGroup in the component palette, you should do all of the following EXCEPT:
            ✔️Panic!
            Learn more about the component through Material Design guides.
            Do a search for "android componentName example".
            Visit the component’s Android documentation.
            Drag it into a layout and see what happens.