⭐❇️➡️✔️
Android Studio Logging
        Printing information to a console (logging) is essential for developers, learn how to use ⭐Android's logging tool, Logcat⭐

    Logging
        To be a great developer, we need to cultivate a “debugging mindset” as well as build up defensive programming practices that make writing error-prone code less likely. When developing apps, we often find ourselves in the role of a coding investigator—hunting for clues and answers to why our app is crashing or not working as expected.

        The Logcat window in Android Studio displays messages to developers—both errors in the code or information and messages we write ourselves using the Log class. It displays messages in real-time and keeps a history so we can view older messages. Most Android developers depend heavily on logcat during development to debug their applications.
            ➡️https://www.youtube.com/watch?v=ezoAGY9Z_HU

    Write log messages
        The Log class allows us to create log messages that appear in Logcat. Generally, we should use the following log methods, listed in order from the highest to the lowest priority.
            ❇️Log.e(String, String) // Error
            ❇️Log.w(String, String) // Warning
            ❇️Log.i(String, String) // Information
            ❇️Log.d(String, String) // Debug
            ❇️Log.v(String, String) // Verbose
        
        For each Log method, the first parameter is a unique tag and the second parameter is the message. The tag of a system log message is a short string indicating the system component from which the message originates. Your tag can be any string that you find helpful, such as the name of the current class.
        
            private static final String TAG = "MainActivity";
            ...
            Log.i(TAG, "Greetings from Main Activity");
        In the example above, our tag is the string "MainActivity"—named after the class it is found in. The message is the string "Greetings from Main Activity".

    Set the log level
        We can control the number of messages that appear by setting the log level. We can display all the messages, or just the messages indicating a level of severity of our choice.

        Logcat will continue to capture all messages regardless of the log level settings we choose to see displayed.

        In the Log level dropdown we may select the following values:
            ❇️Verbose: Show all log messages (the default).
            ❇️Debug: Show debug log messages that are useful during development only, as well as the message levels lower in this list.
            ❇️Info: Show expected log messages for regular usage, as well as the message levels lower in this list.
            ❇️Warn: Show possible issues that are not yet errors, as well as the message levels lower in this list.
            ❇️Error: Show issues that have caused errors, as well as the message level lower in this list.
            ❇️Assert: Show issues that the developer expects should never happen.

    Log message format
        The loge message format is:
            date time PID-TID/package priority/tag: message

        For example::
            2020-01-13 16:57:57.124 21711-21711/com.example.helloworld I/MainActivity: Main Activity

        date: 2020-01-13
        time: 16:57:57
        PID-TID (Process Identifier - Thread Identifier): 21711-21711
        package: com.example.helloworld
        priority: I
        tag: MainActivity
        message: Main Activity


        Imagine that we have the following activity in a MyActivity.java file:
            package com.example.helloworld;
            
            import androidx.appcompat.app.AppCompatActivity;
            
            import android.os.Bundle;
            import android.util.Log;
            
            public class MyActivity extends AppCompatActivity {
                @Override
                protected void onCreate(Bundle savedInstanceState) {
                    super.onCreate(savedInstanceState);
                    setContentView(R.layout.activity_main);
                }
            }

        Fill in the blanks to create a log method of warning that prints out a tag indicating the system component from which the message originates with the message “My Activity”.
            public class MyActivity extends AppCompatActivity {
                private static final ⭐String TAG⭐ = ⭐"MyActivity⭐ 
                @Override
                protected void onCreate(Bundle savedInstanceState) {
                    super.onCreate(savedInstanceState);
                    setContentView(R.layout.activity_main);
                    ⭐Log.w⭐(TAG, ⭐"My Activity⭐ );
                }
            }

        Match the log value with the description:
            ⭐⭐: Show all log messages (the default).
            ⭐⭐: Show debug log messages that are useful during development only, as well as the message levels lower in this list.
            ⭐⭐: Show expected log messages for regular usage, as well as the message levels lower in this list.
            ⭐⭐: Show possible issues that are not yet errors, as well as the message levels lower in this list.
            ⭐⭐: Show issues that have caused errors, as well as the message level lower in this list.
            ⭐⭐: Show issues that the developer expects should never happen.
        Error
        Info
        Warn
        Debug
        Verbose
        Assert