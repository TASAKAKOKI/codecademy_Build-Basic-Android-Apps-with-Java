Starting a Project
        Learn how to start a new Android project

Our First Android App
        Let’s get started on our first app! It’s a programming tradition for our first app in any language or framework to simply say “Hello World”. We will quickly build an Android app that does just that.

        Android Studio does lots of work behind the scenes to make easy work of starting a new project.

        Note: Android Studio updates regularly, so your wizard may look slightly different from what we are showing you. This is usually not a problem; the choices should be similar. Android Studio has auto-generated a Package name based on the information entered. This takes our Application name, eg. Hello World, and our Company name, eg. example.com, and produces a package name from them (com.example.helloworld).

        The Android SDK is the collection of packages of code used to develop apps. The Android SDK is regularly updated—with newer versions, fewer devices are compatible but developers have access to more exciting features.
        ➡️https://www.youtube.com/watch?v=7zWmQhIXGWA

        Review:
            ❇️There are a variety of template options for starting points for Android Studio projects.
            ❇️Any project can be built up from the empty activity template.
            ❇️Selecting the API level is a tradeoff between having access to newer features and less device coverage.
            ❇️The options chosen during the project configuration will be used by Android Studio to generate our app.
            ❇️In the project configurations, we configure the app by saying what you want to call it, what company domain to use, and where you would like to store the files. Android Studio uses the company domain and application name to form the name of the package that will be used for your app.
            ❇️Android versions have a version number and a codename. The version number gives the precise version of Android (e.g., 5.0), while the codename is a slightly more generic “friendly” name that may cover several versions of Android (e.g., Lollipop).

        If our Application name was Most Amazing App and our Company name was codecademy.com what Package name would be generated?
            mostamazingapp.com.codecademy
            com.codecademy.MostAmazingApp
            ➡️com.codecademy.mostamazingapp
            codecademy.com.mostamazingapp

        Which device will have greater coverage?
            ➡️API 21: Android 5.0 (Lollipop)
            API 28: Android 9.0 (Pie)