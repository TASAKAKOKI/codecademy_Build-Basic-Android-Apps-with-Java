Navigating the File Structure
        Android Studio generates a lot of files automatically—let's learn how to navigate them.
    
Generated Code
        We have a working Android development environment! Android Studio generates a lot of code for us when we start a new project. Understanding the architecture it provides—the structure of nested folders and files—helps us navigate our projects so that we can build out our app. Let’s explore this generated code so that we can learn how to build on this empty activity template.
            ➡️https://www.youtube.com/watch?v=88dn6cugvrE
    
        ❇️app > java > com.example.helloworld > MainActivity
            This is the main activity. It is the ⭐entry point for our app⭐. When we build and run our app, ⭐the system launches an instance of this Activity and loads its layout⭐.

        ❇️app > res
            The res folder ⭐contains application resources, such as drawable files, layout files, and UI strings⭐. It is used to store the values for the resources that are used in many Android projects to include features of color, styles, dimensions, etc. The benefit of having these resources separate from our Java code is so that they can be updated independently.

        ❇️app > res > layout > activity_main.xml
            This ⭐XML file defines the layout for the activity’s user interface(UI)⭐. It currently contains a TextView element with the text “Hello World!”

        ❇️app > manifests > AndroidManifest.xml
            The manifest file ⭐describes the fundamental characteristics of the app and defines each of its components⭐.
                ❇️ Android Studio creates this file for us when we create our App.
                ❇️Every Android app must include a file precisely named AndroidManifest.XML at the root of the project source set.
                ❇️The manifest file describes essential information about our app to the Android build tools, the Android operating system, and Google Play.

        ❇️Gradle Scripts > build.gradle
            There are two files with this name: one for the project and one for the app module. Each module has its own build.gradle file, but this project currently has just one module.

            When we click the run button in Android Studio, here are some of the things that Gradle does:
                ❇️Locates and downloads the correct versions of any third-party libraries we need.
                ❇️Calls the correct build tools in the correct sequence to turn all of our source code and resources into a deployable app.
                ❇️Installs and runs our app on an Android device.
        
        The res folder does not contain which of the following?
            Layout files
            UI strings
            Drawable files
            ✔️Java code

        The project-level build.gradle file will override the content in the application module build.gradle file.
            ✔️False
            True
            ➡️The application module build.gradle file will override the top-level build.gradle file.

        In the MainActivity.java file which line of code specifies which layout to use?
            package com.example.helloworld;
            
            import androidx.appcompat.app.AppCompatActivity;
            import android.os.Bundle;
            
            public class MainActivity extends AppCompatActivity {
                @Override
                protected void onCreate(Bundle savedInstanceState) {
                    super.onCreate(savedInstanceState);
                    setContentView(R.layout.activity_main);
                }
            }
                ➡️setContentView(R.layout.activity_main);
            
                import androidx.appcompat.app.AppCompatActivity;
                import android.os.Bundle;

                package com.example.helloworld;

                public class MainActivity extends AppCompatActivity {
                ...
                }