⭐❇️➡️✔️
Game Logic Pt. I
    By this point, our hero (you!) has already defined the Question object, the primary data structure required in Unquote. In today’s episode, the hero will have to keep their wits about them as they build the logic to support their new entertaining endeavor!

    You’ll be adding critical elements to the Question class. After this, Unquote will be able to check whether the player answered each question correctly, generate a random number to help choose the next question, and produce a game over message.

instruction
    1.
    Define an isCorrect() method.

    We want to add a method that returns true if the player answered the question correctly and false otherwise.

    Begin by defining a method in Question.java named, isCorrect().
        Here’s the structure for a Java method named getNumChocoTacosRemaining() within the IceCreamTruck class:
            public class IceCreamTruck {
                int chocoTacoCount;    
                int getNumChocoTacosRemaining() {
                    return chocoTacoCount;
                }
            }
        Methods begin with a return type (int for integer in the case above), followed by a name, getNumChocoTacosRemaining, then optional parameters within the parentheses, and then the code begins within two curly braces { and }.

    2.
    Determine whether the player answered the question correctly.

    How do we know if the player selected the correct answer? Remember that your Question object has the member variables correctAnswer and playerAnswer—both integers.

    The isCorrect() method should compare correctAnswer and playerAnswer. If they’re the same, the method should return true. Otherwise, it should return false.
        You can execute this conditional check in an if statement, on a single line, or assign the comparison to a variable.

        Use the equals comparison operator (==) to determine if one simple data type is equal to another.

        The class IceCreamTruck needs a method lastTaco() that returns true if there’s exactly 1 taco remaining, and false otherwise… Here are a few ways we could write that method:
            public class IceCreamTruck {
                int chocoTacoCount;
                boolean lastTaco() {
                    boolean oneTacoRemaining = chocoTacoCount == 1;
                    return oneTacoRemaining;
                }
            }
        Or on one line:
            public class IceCreamTruck {
                int chocoTacoCount;
                boolean lastTaco() {
                    return chocoTacoCount == 1;
                }
            }

    3.
    MainActivity.java.

    As you’ll see in future lessons, Android applications present a user interface where touch input is processed by Activity objects.

    In Unquote, MainActivity (which inherits from the Activity class) is responsible for presenting the game screen and responding to the player’s choices; therefore, it is largely responsible for the game’s logic.

    MainActivity.java is where you will write the bulk of Java code for this game.

    4.
    Define a generateRandomNumber() method in MainActivity.

    Inside the MainActivity class, define a new method named generateRandomNumber() that returns an integer.

    You will use this method to pick new questions at random to keep Unquote players on their toes!

    5.
    Add a parameter to generateRandomNumber().

    This method will return a random number between 0 and one less than the number of questions remaining in the game, which means we need to provide a maximum upper limit.

    Add an integer parameter to generateRandomNumber() named max.

    6.
    Use Math.random() to generate a random decimal value.

    All Java objects have access to the built-in Math object.

    And Math provides a method named random() which returns a randomly chosen double value greater than or equal to 0, but less than 1.

    For example, some possible return values are 0.2312, 0, or 0.999999.

    Call Math.random() within generateRandomNumber() and save it to a local variable.

    7.
    Calculate a random number between 0 and max

    Use the result from task 6 to calculate a random number between 0 and max (the parameter you pass into generateRandomNumber()) and save it to a local double variable.

    8.
    Cast the result to an integer.

    Ultimately, we need a whole number to decide which question to present next because presenting question number 3.2312 or question 0.7 is impossible (we can’t show 70% of a question… right? 🤔).

    Instead, we will use a programming technique called casting to convert one data type to another, in this case, from a double type to an int type.

    An integer cannot retain any decimal values, so given the doubles above, 3.2312 casts down to 3, and 0.7 casts down to 0. In both cases, we lose the decimal value.

    To cast, we place the resulting type we desire as a prefix to the original data (in parentheses). Check out this example of casting from an int to a boolean:
        int timmysDollars = 5;
        // The following line casts timmysDollars into a boolean (non-zero values become true, 0 becomes false)
        boolean timmyHasMoney = (boolean) timmysDollars;
        if (timmyHasMoney) {
            System.out.println("Fresh Choco Tacos, Timmy — have at 'em!");
        } else {
            System.out.println("Scram, kid! I've got bills to pay!");
        }

    Within generateRandomNumber(), cast the result from task 7 to an integer type, and save it to a new variable.

    9.
    Return the randomly-generated number.

    Within generateRandomNumber(), return the value calculated in the previous step.

    10.
    Game. Over.

    When the player submits their answer to the final question, the game ends. At which point, you present the player with one of two possible messages:

    “You got all 6 right! You won!”
    Or “You got 3 right out of 6. Better luck next time!”
    You will create a method that generates this String message.

    Begin by defining a method in MainActivity named getGameOverMessage(), it must accept two integer inputs (totalCorrect and totalQuestions), and return a String object.

    11.
    Use an if / else statement to decide which message to create.

    Set up an if / else statement that compares totalCorrect with totalQuestions.

    When the two values are equal, you will build and return String 1, if not, you will build and return String 2.

    An if statement begins with a conditional check in parentheses (a check to see whether a statement is true or false), and the code proceeding the check executes only if the condition evaluates to true.

    12.
    Create two String objects.

    Inside your if / else statement, create the two String objects you need.

    13.
    Return the expected String from getGameOverMessage().

    Previously, you have encountered one return statement per method, and usually at the bottom.

    However, it is possible to code multiple return statements in the same method body.

    Use one or more return statements to return the correct String back from your getGameOverMessage() method.

    14.
    Test out your new game logic!

    In Main.java, we’ve provided some code to test out your new methods.

    Run the Main.java file to make sure isCorrect(), generateRandomNumber(), and getGameOverMessage() behave exactly as you expect.